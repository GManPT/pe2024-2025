\documentclass[a4paper,12pt]{article}
\usepackage{titling}   
\usepackage[portuguese]{babel}
\usepackage{amsmath} 
\usepackage{graphicx}
\usepackage{xcolor}
\usepackage{amssymb}
\usepackage{tikz}
\usepackage{cancel}
\usepackage{array}
\usepackage{booktabs}
\usepackage{mdframed}
\usepackage{enumitem}
\usetikzlibrary{automata, positioning}

\setlength{\droptitle}{-6em}
\pretitle{\begin{center}\LARGE}
\posttitle{\par\end{center}}
\preauthor{\begin{center}\large}
\postauthor{\end{center}} 
\predate{\begin{center}\large} 
\postdate{\end{center}} 

\author{Series de Problemas 4}

\begin{document}

\title{Probabilidade e Estatística}
\date{08 de Maio de 2025}
\maketitle

\begin{enumerate}
    \item \textbf{Admita que o número de programas examinados de modo independente até que se observe o primeiro programa que não compile é representado pela variável aleatória \( X \) com distribuição geométrica com parâmetro \( p \), onde \( p \) é uma probabilidade desconhecida. Determine a estimativa de máxima verosimilhança de \( p \), atendendo à amostra \( (4, 50, 5, 13, 30) \) proveniente da população \( X \).}

    \vspace{0.3cm}

    \begin{mdframed}[backgroundcolor=gray!10, linewidth=0pt, innertopmargin=10pt, innerbottommargin=10pt]
    \textbf{Resolução:}

    Considere-se a variável aleatória $X \sim \text{Geom}(p)$ que representa o número de programas examinados até observar o primeiro que não compile, incluindo este. A função de probabilidade da distribuição geométrica é:
    \begin{align*}
    P(X = x) = (1 - p)^{x - 1} p, \quad x = 1, 2, 3, \ldots
    \end{align*}

    Para a amostra $(x_1, x_2, x_3, x_4, x_5) = (4, 50, 5, 13, 30)$, a função de verosimilhança é:
    \begin{align*}
    L(p) &= \prod_{i=1}^5 P(X = x_i) \\
    &= \prod_{i=1}^5 (1 - p)^{x_i - 1} p \\
    &= p^5 \times (1 - p)^{\sum_{i=1}^5 (x_i - 1)} \\
    &= p^5 \times (1 - p)^{97}
    \end{align*}

    Para maximizar a verosimilhança, aplique-se o logaritmo natural:
    \begin{align*}
    \ell(p) &= \log L(p) \\
    &= \log \left( p^5 \times (1 - p)^{97} \right) \\
    &= 5 \log p + 97 \log(1-p)
    \end{align*}

    Derive-se $\ell(p)$ em relação a $p$:
    \begin{align*}
    \frac{d\ell}{dp} &= \frac{d}{dp} \left( 5 \log p + 97 \log(1-p) \right) \\
    &= \frac{5}{p} + 97 \cdot \frac{1}{1-p} \cdot (-1) \\
    &= \frac{5}{p} - \frac{97}{1-p}
    \end{align*}

    Igualando a derivada a zero para encontrar o valor de $p$ que maximiza $\ell(p)$:
    \begin{align*}
    \frac{5}{p} - \frac{97}{1-p} &= 0 \\
    \frac{5}{p} &= \frac{97}{1-p} \\
    5(1-p) &= 97p \\
    5 - 5p &= 97p \\
    5 &= 102p \\
    p &= \frac{5}{102} \approx 0.0490196
    \end{align*}

    \textbf{Resposta:} A estimativa de máxima verosimilhança de $p$ é $\frac{5}{102} \approx 0.0490196$.
    \end{mdframed}

    \vspace{0.5cm}

    \item \textbf{Tem-se assumido que o impacto hidrodinâmico (com valores medidos em unidades apropriadas), \( X \) do casco de um navio sobre uma onda em determinada região do globo possui função de densidade de probabilidade dada por}
    \[
    f(x) = \left| \frac{2x}{\lambda^2} \right| e^{-\frac{x^2}{\lambda^2}},
    \]
    \textbf{onde \( x > 0 \) e o parâmetro \( \lambda \) é uma constante positiva desconhecida. Tendo por base uma amostra aleatória \( (X_1, X_2, \ldots, X_{13}) \) de \( X \) com \( n = 13 \), determine a estimativa de máxima verosimilhança do parâmetro \( \lambda \) para uma realização da amostra tal que \( \sum_{i=1}^{13} x_i^2 = 62 \).}

    \vspace{0.3cm}

    \begin{mdframed}[backgroundcolor=gray!10, linewidth=0pt, innertopmargin=10pt, innerbottommargin=10pt]
    \textbf{Resolução:}

    Comece-se por observar que a função de densidade fornecida pode ser reescrita, já que \( x > 0 \), como:
    \[
    f(x) = \frac{2x}{\lambda^2} e^{-\frac{x^2}{\lambda^2}}, \quad x > 0
    \]

    Trata-se da distribuição de Rayleigh generalizada. A função de verosimilhança para uma amostra \( (x_1, x_2, \ldots, x_n) \) é dada por:
    \begin{align*}
    L(\lambda) &= \prod_{i=1}^n \frac{2x_i}{\lambda^2} e^{-\frac{x_i^2}{\lambda^2}} \\
    &= \left( \frac{2^n}{\lambda^{2n}} \prod_{i=1}^n x_i \right) \cdot e^{-\frac{1}{\lambda^2} \sum_{i=1}^n x_i^2}
    \end{align*}

    Passe-se à log-verosimilhança:
    \begin{align*}
    \ell(\lambda) &= \log L(\lambda) \\
    &= n \log 2 - 2n \log \lambda + \sum_{i=1}^n \log x_i - \frac{1}{\lambda^2} \sum_{i=1}^n x_i^2
    \end{align*}

    Derivando em relação a \( \lambda \) e igualando a zero:
    \begin{align*}
    \frac{d\ell}{d\lambda} &= -\frac{2n}{\lambda} + \frac{2}{\lambda^3} \sum_{i=1}^n x_i^2 = 0 \\
    \Rightarrow -2n\lambda^2 + 2 \sum_{i=1}^n x_i^2 &= 0 \\
    \Rightarrow \lambda^2 &= \frac{1}{n} \sum_{i=1}^n x_i^2 \\
    \Rightarrow \hat{\lambda} &= \sqrt{\frac{1}{n} \sum_{i=1}^n x_i^2}
    \end{align*}

    Apliquemos agora à amostra fornecida:
    \begin{align*}
    n &= 13 \\
    \sum_{i=1}^n x_i^2 &= 62 \\
    \hat{\lambda} &= \sqrt{\frac{62}{13}} = \sqrt{4.7692} \approx 2.1839
    \end{align*}

    \textbf{Resposta:} A estimativa de máxima verosimilhança de \( \lambda \) é \( 2{,}18 \) (com duas casas decimais).
    \end{mdframed}

    \vspace{0.5cm}

    \item \textbf{Considere a variável aleatória \( X \sim \text{Poi}(\lambda) \), que modela o número de participações de sinistros automóveis a determinada seguradora num período de uma hora, e uma amostra aleatória \((X_1, X_2, \ldots, X_n)\) de \(X\). Calcule a estimativa de máxima verosimilhança da probabilidade de ocorrerem mais de 3 participações de sinistros automóveis às seguradoras numa hora, sabendo que a concretização de uma amostra aleatória de dimensão 22 de \(X\) conduziu a}  
    \[
    \sum_{i=1}^{22} x_i = 42.
    \]

    \begin{enumerate}[label=\alph*)]
        \item \textbf{0.126856}
        \item \textbf{0.873144}
        \item \textbf{0.002071}
        \item \textbf{0.983743}
    \end{enumerate}

    \vspace{0.3cm}

    \begin{mdframed}[backgroundcolor=gray!10, linewidth=0pt, innertopmargin=10pt, innerbottommargin=10pt]
    \textbf{Resolução:}

    Comece-se por recordar que a esperança da distribuição de Poisson é dada por \( E(X) = \lambda \). Assim, a estimativa de máxima verosimilhança (EMV) de \( \lambda \), com base na amostra de dimensão \( n = 22 \) e soma dos valores observados igual a 42, é:
    \[
    \hat{\lambda} = \frac{1}{n} \sum_{i=1}^{n} x_i = \frac{42}{22} \approx 1.9091
    \]

    Pretende-se estimar a probabilidade de ocorrerem mais de 3 sinistros numa hora, ou seja:
    \[
    P(X > 3) = 1 - P(X \leq 3)
    \]

    Com \( X \sim \text{Poi}(\hat{\lambda}) = \text{Poi}(1.9091) \), utilizando a função \texttt{PoissonCD} da calculadora Casio fx-CG50, obtém-se:
    \[
    P(X \leq 3) \approx 0.8731
    \]

    Portanto:
    \[
    P(X > 3) = 1 - 0.8731 = 0.1269
    \]

    Arredondando com seis casas decimais, temos:
    \[
    \boxed{0.126856}
    \]

    \textbf{Resposta correta:} a) \textbf{0.126856}
    \end{mdframed}

    \vspace{0.5cm}

    \item \textbf{Admita que o tempo de vida em centenas de horas, \( X \), de um novo tipo de lâmpadas de longa duração segue uma distribuição exponencial cujo parâmetro \( \lambda \) é uma constante desconhecida e positiva. Com o objetivo de estimar \( \lambda \), registaram-se as durações de lâmpadas deste tipo, tendo-se obtido a seguinte amostra: \( (91, 81, 102, 72, 120) \). Calcule a estimativa de máxima verosimilhança da probabilidade de uma lâmpada desse tipo durar mais de 50 centenas de horas.}

    \vspace{0.3cm}

    \begin{enumerate}[label=\alph*)]
        \item \textbf{0.4922}
        \item \textbf{0.7504}
        \item \textbf{0.5848}
        \item \textbf{0.6758}
    \end{enumerate}

    \vspace{0.3cm}

    \begin{mdframed}[backgroundcolor=gray!10, linewidth=0pt, innertopmargin=10pt, innerbottommargin=10pt]
    \textbf{Resolução:}

    Comecemos por lembrar que a função de densidade da distribuição exponencial é dada por:
    \[
    f(x; \lambda) = \lambda e^{-\lambda x}, \quad x > 0
    \]

    A função de verosimilhança para uma amostra \( (x_1, x_2, \ldots, x_n) \) é:
    \begin{align*}
    L(\lambda) &= \prod_{i=1}^n \lambda e^{-\lambda x_i} \\
    &= \lambda^n e^{-\lambda \sum_{i=1}^n x_i}
    \end{align*}

    Passando à log-verosimilhança:
    \begin{align*}
    \ell(\lambda) &= \log L(\lambda) \\
    &= n \log \lambda - \lambda \sum_{i=1}^n x_i
    \end{align*}

    Derivando em relação a \( \lambda \) e igualando a zero:
    \begin{align*}
    \frac{d\ell}{d\lambda} &= \frac{n}{\lambda} - \sum_{i=1}^n x_i = 0 \\
    \Rightarrow \frac{n}{\lambda} &= \sum_{i=1}^n x_i \\
    \Rightarrow \hat{\lambda} &= \frac{n}{\sum_{i=1}^n x_i}
    \end{align*}

    Note-se que esta é a expressão do estimador de máxima verosimilhança para o parâmetro \( \lambda \) da distribuição exponencial. Alternativamente, sabendo que \( E(X) = \frac{1}{\lambda} \), temos \( \hat{\lambda} = \frac{1}{\bar{x}} \).

    Dada a amostra \( (91, 81, 102, 72, 120) \), com \( n = 5 \), calcule-se:
    \begin{align*}
    \sum_{i=1}^n x_i &= 91 + 81 + 102 + 72 + 120 = 466 \\
    \hat{\lambda} &= \frac{5}{466} \approx 0.0107296
    \end{align*}

    Para uma variável aleatória exponencial, a probabilidade de ultrapassar um valor \( t \) é dada por:
    \begin{align*}
    P(X > t) &= e^{-\lambda t}
    \end{align*}

    Assim, a estimativa de máxima verosimilhança da probabilidade de uma lâmpada durar mais de 50 centenas de horas é:
    \begin{align*}
    P(X > 50) &= e^{-\hat{\lambda} \cdot 50} \\
    &= e^{-0.0107296 \cdot 50} \\
    &= e^{-0.53648} \\
    &\approx 0.5848
    \end{align*}

    \textbf{Resposta correta:} c) \textbf{0.5848}
    \end{mdframed}

    \vspace{0.5cm}
    
    \item \textbf{Admita que a proporção de potássio em certo fertilizante é representada pela variável aleatória \( X \) com função de densidade de probabilidade}
    \[
    f_X(x) = 
    \begin{cases} 
    \theta(1 - x)^{\theta-1} & 0 < x < 1 \\ 
    0 & \text{caso contrário,}
    \end{cases}
    \]
    \textbf{onde \( \theta \) é um parâmetro positivo desconhecido. Determine a estimativa de máxima verosimilhança de \( P(X \leq 0{,}12) \) baseada na amostra \( (0.666, 0.117, 0.167, 0.57, 0.106, 0.095) \) proveniente da população \( X \).}

    \vspace{0.3cm}

    \begin{mdframed}[backgroundcolor=gray!10, linewidth=0pt, innertopmargin=10pt, innerbottommargin=10pt]
    \textbf{Resolução:}

    Observe-se que a função densidade corresponde a uma distribuição Beta(1, \( \theta \)), também conhecida como distribuição de potência. A função de verosimilhança para a amostra \( (x_1, x_2, \dots, x_n) \) é dada por:
    \begin{align*}
    L(\theta) &= \prod_{i=1}^{n} \theta (1 - x_i)^{\theta - 1} \\
    &= \theta^n \cdot \prod_{i=1}^{n} (1 - x_i)^{\theta - 1}
    \end{align*}

    Tomando o logaritmo para facilitar a maximização:
    \begin{align*}
    \ell(\theta) &= \log L(\theta) \\
    &= n \log \theta + (\theta - 1) \sum_{i=1}^{n} \log(1 - x_i)
    \end{align*}

    Derivando em relação a \( \theta \) e igualando a zero:
    \begin{align*}
    \frac{d\ell}{d\theta} &= \frac{n}{\theta} + \sum_{i=1}^{n} \log(1 - x_i) = 0 \\
    \Rightarrow \frac{n}{\theta} &= -\sum_{i=1}^{n} \log(1 - x_i) \\
    \Rightarrow \hat{\theta} &= -\frac{n}{\sum_{i=1}^{n} \log(1 - x_i)}
    \end{align*}

    Com a amostra fornecida \( (0.666, 0.117, 0.167, 0.57, 0.106, 0.095) \), calcule-se:
    \begin{align*}
    \sum_{i=1}^{n} \log(1 - x_i) &= \log(1-0.666) + \log(1-0.117) + \log(1-0.167) \\
    &+ \log(1-0.57) + \log(1-0.106) + \log(1-0.095) \\
    &= \log(0.334) + \log(0.883) + \log(0.833) \\
    &+ \log(0.430) + \log(0.894) + \log(0.905) \\
    &\approx -1.0966 -0.1244 -0.1827 -0.8440 -0.1120 -0.0998 \\ % Valores individuais corrigidos (arred. 4 c.d.)
    &= -2.4596 % Soma confirmada
    \end{align*}

    Assim, a estimativa de máxima verosimilhança do parâmetro é:
    \begin{align*}
    \hat{\theta} &= -\frac{6}{-2.4596} \\
    &\approx 2.4394
    \end{align*}

    Agora, para calcular \( P(X \leq 0.12) \) com o parâmetro estimado:
    \begin{align*}
    P(X \leq 0.12) &= \int_0^{0.12} \hat{\theta}(1 - x)^{\hat{\theta} - 1} dx
    \end{align*}

    Utilizando a substituição \( u = 1 - x \Rightarrow du = -dx \), mudando os limites:
    \begin{align*}
    x = 0 &\Rightarrow u = 1 \\
    x = 0.12 &\Rightarrow u = 0.88
    \end{align*}

    Com esta substituição, o integral torna-se:
    \begin{align*}
    P(X \leq 0.12) &= \int_{0.88}^{1} \hat{\theta} u^{\hat{\theta} - 1} (-du) \\
    &= \int_{1}^{0.88} -\hat{\theta} u^{\hat{\theta} - 1} du \\
    &= \left[ -u^{\hat{\theta}} \right]_{1}^{0.88} \\
    &= -0.88^{\hat{\theta}} - (-1^{\hat{\theta}}) \\
    &= 1 - 0.88^{2.44} \\
    &\approx 1 - 0.7321 \\
    &= 0.2679
    \end{align*}

    \textbf{Resposta:} A estimativa de máxima verosimilhança de \( P(X \leq 0{,}12) \) é aproximadamente 0.2679.
    \end{mdframed}

\end{enumerate}

\end{document}