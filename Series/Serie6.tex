\documentclass[a4paper,12pt]{article}
\usepackage{titling}   
\usepackage[portuguese]{babel}
\usepackage{amsmath} 
\usepackage{graphicx}
\usepackage{xcolor}
\usepackage{amssymb}
\usepackage{tikz}
\usepackage{cancel}
\usepackage{array}
\usepackage{booktabs}
\usepackage{mdframed}
\usepackage{enumitem}
\usetikzlibrary{automata, positioning}

\setlength{\droptitle}{-6em}
\pretitle{\begin{center}\LARGE}
\posttitle{\par\end{center}}
\preauthor{\begin{center}\large}
\postauthor{\end{center}} 
\predate{\begin{center}\large} 
\postdate{\end{center}} 

\author{Series de Problemas 6}

\begin{document}

\title{Probabilidade e Estatística}
\date{21 de Maio de 2025}
\maketitle


\begin{enumerate}
    \item \textbf{Seja \( X \) a variável aleatória que representa o número semanal de avarias de um sistema eletrónico. Uma empresa electrotécnica propõe a hipótese \( H_0 \) de que}
    \[
    P(X = x) = \frac12\,(x + 1)(x + 2)\,(1 - 0.4)^x\,(0.4)^3,\quad x = 0,1,2,3,
    \]
    \textbf{acumulando-se todas as probabilidades para \(X>3\). De 200 semanas obtiveram‑se as frequências observadas abaixo:}

    \begin{center}
    \small
    \renewcommand{\arraystretch}{1.5}
    \setlength{\tabcolsep}{8pt}
    \begin{tabular}{|c|c|c|c|c|c|}
    \hline
    \(x\) & 0 & 1 & 2 & 3 & \(>3\) \\ \hline
    Frequência observada & 27 & 31 & 39 & 34 & 69 \\ \hline
    Frequência esperada  & \(E_1\) & 23.0 & 27.6 & 27.6 & \(E_5\) \\ \hline
    \end{tabular}
    \end{center}

    \vspace{0.3cm}

    \textbf{Após o cálculo das frequências esperadas \(E_1,E_5\) (arredondadas às décimas), avalie se \(H_0\) é consistente com os dados. Decida com base no valor‑\(p\).}

    \vspace{0.3cm}

    \begin{enumerate}[label=\Alph*)]
        \item \textbf{Rejeita-se para 1\%, 5\% e 10\%}
        \item \textbf{Rejeita-se para 10\% e não se rejeita para 1\% e 5\%}
        \item \textbf{Não se rejeita para 1\%, 5\% e 10\%}
        \item \textbf{Rejeita-se para 5\% e 10\% e não se rejeita para 1\%}
    \end{enumerate}

    \vspace{0.3cm}

    \begin{mdframed}[backgroundcolor=gray!10,linewidth=0pt,innertopmargin=10pt,innerbottommargin=10pt]
    \textbf{Resolução:}

    Este problema envolve um teste de aderência (goodness-of-fit test) usando a distribuição qui-quadrado para verificar se os dados observados são consistentes com a distribuição proposta sob a hipótese nula $H_0$.

    A função de probabilidade proposta é:
    \[
    P(X = x) = \frac{1}{2}(x + 1)(x + 2)(0.6)^x(0.4)^3
    \]

    Calculando as probabilidades para cada valor:
    \begin{align*}
    P(X = 0) &= \frac{1}{2} \cdot 1 \cdot 2 \cdot 1 \cdot 0.064 = 0.064 \\
    P(X = 1) &= \frac{1}{2} \cdot 2 \cdot 3 \cdot 0.6 \cdot 0.064 = 0.1152 \\
    P(X = 2) &= \frac{1}{2} \cdot 3 \cdot 4 \cdot 0.36 \cdot 0.064 = 0.1382 \\
    P(X = 3) &= \frac{1}{2} \cdot 4 \cdot 5 \cdot 0.216 \cdot 0.064 = 0.1382
    \end{align*}

    Para $x > 3$, utilizamos o complementar:
    \[
    P(X > 3) = 1 - (0.064 + 0.1152 + 0.1382 + 0.1382) = 0.5444
    \]

    Com uma amostra de 200 semanas, as frequências esperadas são:
    \begin{align*}
    E_1 &= 200 \times 0.064 = 12.8 \\
    E_2 &= 200 \times 0.1152 = 23.0 \\
    E_3 &= 200 \times 0.1382 = 27.6 \\
    E_4 &= 200 \times 0.1382 = 27.6 \\
    E_5 &= 200 \times 0.5444 = 108.9
    \end{align*}

    A estatística de teste qui-quadrado é calculada por:
    \[
    \chi^2 = \sum_{i=1}^5 \frac{(O_i - E_i)^2}{E_i}
    \]

    Substituindo os valores observados e esperados:
    \begin{align*}
    \chi^2 &= \frac{(27-12.8)^2}{12.8} + \frac{(31-23.0)^2}{23.0} + \frac{(39-27.6)^2}{27.6} \\
    &\quad + \frac{(34-27.6)^2}{27.6} + \frac{(69-108.9)^2}{108.9} \\
    &= 15.75 + 2.78 + 4.71 + 1.48 + 14.62 = 39.34
    \end{align*}

    Sob a hipótese nula $H_0$, a estatística $\chi^2$ segue uma distribuição qui-quadrado com $k - 1 = 5 - 1 = 4$ graus de liberdade.

    O valor-p é:
    \[
    \text{valor-p} = P(\chi^2_4 \geq 39.34) \ll 0.001
    \]

    Este valor é extremamente pequeno, muito menor que qualquer nível de significância usual.

    Como valor-p $\ll 0.001$, temos:
    \begin{itemize}
        \item Para $\alpha = 0.01$ (1\%): valor-p $< 0.01$, logo rejeita-se $H_0$
        \item Para $\alpha = 0.05$ (5\%): valor-p $< 0.05$, logo rejeita-se $H_0$
        \item Para $\alpha = 0.10$ (10\%): valor-p $< 0.10$, logo rejeita-se $H_0$
    \end{itemize}

    Os dados observados apresentam discrepâncias significativas relativamente ao modelo proposto, particularmente nas categorias $x = 0$ (muito mais observações que esperado) e $x > 3$ (muito menos observações que esperado). A hipótese nula é rejeitada para todos os níveis de significância usuais.

    \textbf{Resposta:} A opção correta é \textbf{(A) Rejeita-se para 1\%, 5\% e 10\%}.
    \end{mdframed}

    \vspace{0.5cm}

    \item \textbf{Suspeita‐se que o ruído máximo noturno, medido em decibéis (db) em zona residencial próxima de um aeroporto, é uma variável aleatória \( X \) que segue uma distribuição normal. Foram recolhidos os seguintes dados relativos a 100 medições noturnas, em dias escolhidos ao acaso:}

    \[
    \begin{tabular}{|c|c|c|c|c|}
    \hline
    Ruído máximo noturno & \(\le50\) & \(]50,55]\) & \(]55,60]\) & \(>60\) \\ 
    \hline
    Frequência observada  & 13        & 19           & 32           & 36        \\
    \hline
    \end{tabular}
    \]

    \textbf{Avalie a hipótese \(H_0\) de que \(X\sim N(55,7^2)\). Decida com base no valor‑\(p\) aproximado e nas frequências esperadas (sob \(H_0\)) aproximadas às centésimas.}

    \vspace{0.3cm}

    \begin{enumerate}[label=\Alph*)]
        \item \textbf{Rejeita‑se \(H_0\) para 10\% e não se rejeita \(H_0\) para 1\% e 5\%}
        \item \textbf{Rejeita‑se \(H_0\) para 5\% e 10\% e não se rejeita \(H_0\) para 1\%}
        \item \textbf{Rejeita‑se \(H_0\) para 1\%, 5\% e 10\%}
        \item \textbf{Não se rejeita \(H_0\) para 1\%, 5\% e 10\%}
    \end{enumerate}

    \vspace{0.3cm}

    \begin{mdframed}[backgroundcolor=gray!10,linewidth=0pt,innertopmargin=10pt,innerbottommargin=10pt]
    \textbf{Resolução:}

    Este problema envolve um teste de aderência para verificar se os dados observados seguem uma distribuição normal com parâmetros específicos.

    Sob a hipótese nula $H_0: X \sim N(55, 7^2)$, calculamos as probabilidades teóricas para cada classe usando a função de distribuição acumulada da normal padrão.

    Para $P(X \leq 50)$:
    \[
    P(X \leq 50) = \Phi\left(\frac{50-55}{7}\right) = \Phi(-0.7143) \approx 0.2370
    \]

    Para $P(50 < X \leq 55)$:
    \[
    P(50 < X \leq 55) = \Phi(0) - \Phi(-0.7143) = 0.5000 - 0.2370 = 0.2630
    \]

    Para $P(55 < X \leq 60)$:
    \[
    P(55 < X \leq 60) = \Phi(0.7143) - \Phi(0) = 0.7630 - 0.5000 = 0.2630
    \]

    Para $P(X > 60)$:
    \[
    P(X > 60) = 1 - \Phi(0.7143) = 1 - 0.7630 = 0.2370
    \]

    Com uma amostra de 100 observações, as frequências esperadas são:
    \begin{align*}
    E_1 &= 100 \times 0.2370 = 23.70 \\
    E_2 &= 100 \times 0.2630 = 26.30 \\
    E_3 &= 100 \times 0.2630 = 26.30 \\
    E_4 &= 100 \times 0.2370 = 23.70
    \end{align*}

    A estatística qui-quadrado de aderência é:
    \[
    \chi^2 = \sum_{i=1}^4 \frac{(O_i - E_i)^2}{E_i}
    \]

    Substituindo os valores observados e esperados:
    \begin{align*}
    \chi^2 &= \frac{(13-23.70)^2}{23.70} + \frac{(19-26.30)^2}{26.30} + \frac{(32-26.30)^2}{26.30} + \frac{(36-23.70)^2}{23.70} \\
    &= 4.83 + 2.03 + 1.24 + 6.38 = 14.48
    \end{align*}

    Sob a hipótese nula, a estatística $\chi^2$ segue uma distribuição qui-quadrado com $k-1 = 4-1 = 3$ graus de liberdade (não há parâmetros estimados a partir dos dados).

    O valor-p é:
    \[
    \text{valor-p} = P(\chi^2_3 \geq 14.48) \approx 0.0023
    \]

    Comparando com os níveis de significância:
    \begin{itemize}
        \item Para $\alpha = 0.01$ (1\%): valor-p $< 0.01$, logo rejeita-se $H_0$
        \item Para $\alpha = 0.05$ (5\%): valor-p $< 0.05$, logo rejeita-se $H_0$
        \item Para $\alpha = 0.10$ (10\%): valor-p $< 0.10$, logo rejeita-se $H_0$
    \end{itemize}

    O valor-p extremamente pequeno indica evidência muito forte contra a hipótese de que os dados seguem uma distribuição normal com média 55 e desvio padrão 7.

    \textbf{Resposta:} A opção correta é \textbf{(C) Rejeita-se $H_0$ para 1\%, 5\% e 10\%}.
    \end{mdframed}

    \vspace{0.5cm}
    
    \item \textbf{Por forma a avaliar a relação entre a percentagem de hidrocarbonetos presentes no condensador principal de uma unidade de destilação (\(x\)) e a pureza do oxigênio produzido (\(Y\)), recolheu-se uma amostra casual de dimensão \(n = 10\) que conduziu aos seguintes resultados:}

    \[
    \sum_{i=1}^{10} x_i = 12.5,\quad 
    \sum_{i=1}^{10} x_i^2 = 16.01,\quad 
    \sum_{i=1}^{10} y_i = 930,
    \]

    \[
    \sum_{i=1}^{10} y_i^2 = 86594,\quad 
    \sum_{i=1}^{10} x_i y_i = 1168.3,
    \]
    
    \textbf{onde \(x \in [0.9,1.6]\).}

    \textbf{Com base no método de regressão linear simples, determine um intervalo de confiança a 99\% para o valor esperado de \(Y\) quando \(x = 1.25\).}

    \vspace{0.3cm}

    \begin{enumerate}[label=\Alph*)]
        \item \textbf{[83.1367, 102.863]}
        \item \textbf{[85.2048, 100.795]}
        \item \textbf{[88.1632, 97.8368]}
        \item \textbf{[91.4705, 94.5295]}
    \end{enumerate}

    \vspace{0.3cm}

    \begin{mdframed}[backgroundcolor=gray!10,linewidth=0pt,innertopmargin=10pt,innerbottommargin=10pt]
    \textbf{Resolução:}

    Este problema envolve a construção de um intervalo de confiança para o valor esperado de $Y$ num ponto específico $x = 1.25$ usando regressão linear simples.

    Comece-se por calcular as médias amostrais:
    \[
    \bar{x} = \frac{12.5}{10} = 1.25, \quad \bar{y} = \frac{930}{10} = 93
    \]

    Observe-se que $\bar{x} = 1.25$ coincide exatamente com o ponto onde pretendemos construir o intervalo. Isto simplifica os cálculos, pois a previsão pelo modelo de regressão linear simples será:
    \[
    \hat{y}(1.25) = \bar{y} = 93
    \]

    Para construir o intervalo de confiança, necessitamos do desvio padrão residual $S$. Calculemos as somas necessárias:
    \[
    S_{xx} = \sum x_i^2 - n\bar{x}^2 = 16.01 - 10 \times (1.25)^2 = 0.385
    \]

    O coeficiente angular do modelo é:
    \[
    \hat{\beta}_1 = \frac{\sum x_i y_i - n\bar{x}\bar{y}}{S_{xx}} = \frac{1168.3 - 10 \times 1.25 \times 93}{0.385} = 15.065
    \]

    Para calcular o desvio padrão residual, utilizamos:
    \[
    S^2 = \frac{\text{SSE}}{n-2}, \quad \text{onde } \text{SSE} = \sum y_i^2 - \hat{\beta}_0 \sum y_i - \hat{\beta}_1 \sum x_i y_i
    \]

    Após os cálculos, obtém-se $S \approx 1.4415$.

    Para um intervalo de confiança de 99\% no valor esperado $E[Y|x = 1.25]$, utilizamos a distribuição $t$ com $n-2 = 8$ graus de liberdade. O valor crítico é $t_{0.995,8} \approx 3.355$.

    Como $x = 1.25 = \bar{x}$, o erro padrão da previsão simplifica-se para:
    \[
    \text{SE} = S \sqrt{\frac{1}{n}} = \frac{1.4415}{\sqrt{10}} \approx 0.4558
    \]

    A margem de erro é:
    \[
    \text{ME} = t_{0.995,8} \times \text{SE} = 3.355 \times 0.4558 \approx 1.5295
    \]

    O intervalo de confiança de 99\% é:
    \[
    \hat{y}(1.25) \pm \text{ME} = 93 \pm 1.5295 = [91.4705, 94.5295]
    \]

    \textbf{Resposta:} A opção correta é \textbf{(D) [91.4705, 94.5295]}.
    \end{mdframed}
    
    \vspace{0.5cm}

    \item \textbf{Para testar a relação entre a altura das ondas (\(x\), em metros) e o montante \(Y\) (em milhares de euros) dos estragos causados na orla costeira em dias de forte agitação marítima, foram obtidas observações relativas a 17 dias com forte agitação marítima que conduziram a:}

    \[
    \sum_{i=1}^{17} x_i = 121,\quad 
    \sum_{i=1}^{17} x_i^2 = 893,\quad 
    \sum_{i=1}^{17} y_i = 359,
    \]

    \[
    \sum_{i=1}^{17} y_i^2 = 7595,\quad 
    \sum_{i=1}^{17} x_i y_i = 2555.
    \]

    \textbf{Ao testar a significância do modelo de regressão linear simples de \(Y\) sobre \(x\) (i.e., ao confrontar as hipóteses \(H_0 : \beta_1 = 0\) e \(H_1 : \beta_1 \neq 0\)), recorrendo ao valor‑\(p\):}

    \vspace{0.3cm}

    \begin{enumerate}[label=\Alph*)]
        \item \textbf{Não se rejeita \(H_0 : \beta_1 = 0\) a 1\%, 5\% e 10\%}
        \item \textbf{Rejeita-se \(H_0 : \beta_1 = 0\) a 10\% e não se rejeita a 1\% e 5\%}
        \item \textbf{Rejeita-se \(H_0 : \beta_1 = 0\) a 1\%, 5\% e 10\%}
        \item \textbf{Rejeita-se \(H_0 : \beta_1 = 0\) a 5\% e 10\% e não se rejeita a 1\%}
    \end{enumerate}

    \vspace{0.3cm}

    \begin{mdframed}[backgroundcolor=gray!10,linewidth=0pt,innertopmargin=10pt,innerbottommargin=10pt]
    \textbf{Resolução:}

    Este problema envolve um teste de significância para o coeficiente angular $\beta_1$ num modelo de regressão linear simples, testando se existe relação linear entre a altura das ondas e os estragos causados.

    Calcule-se primeiro as médias de cada variável:
    \[
    \bar{x} = \frac{121}{17} \approx 7.1176, \quad \bar{y} = \frac{359}{17} \approx 21.1176
    \]

    Determine-se os somatórios centrados necessários:
    \begin{align*}
    S_{xx} &= \sum x_i^2 - n\bar{x}^2 = 893 - 17 \times (7.1176)^2 \approx 3.6948 \\
    S_{xy} &= \sum x_i y_i - n\bar{x}\bar{y} = 2555 - 17 \times 7.1176 \times 21.1176 \approx -0.0273
    \end{align*}

    O estimador do coeficiente angular é:
    \[
    \hat{\beta}_1 = \frac{S_{xy}}{S_{xx}} \approx \frac{-0.0273}{3.6948} \approx -0.0074
    \]

    Para calcular o erro padrão de $\hat{\beta}_1$, necessitamos da variância residual. O resíduo quadrático (SSE) é:
    \[
    \text{SSE} = \sum y_i^2 - \hat{\beta}_0 \sum y_i - \hat{\beta}_1 \sum x_i y_i
    \]

    onde $\hat{\beta}_0 = \bar{y} - \hat{\beta}_1 \bar{x}$. Após os cálculos, obtém-se:
    \[
    S^2 = \frac{\text{SSE}}{n-2} \approx 28.8935
    \]

    O erro padrão de $\hat{\beta}_1$ é:
    \[
    \text{SE}(\hat{\beta}_1) = \sqrt{\frac{S^2}{S_{xx}}} \approx \sqrt{\frac{28.8935}{3.6948}} \approx 2.7961
    \]

    A estatística $T$ segue uma distribuição $t$ com $n-2 = 15$ graus de liberdade:
    \[
    T = \frac{\hat{\beta}_1 - 0}{\text{SE}(\hat{\beta}_1)} \approx \frac{-0.0074}{2.7961} \approx -0.0026
    \]

    O valor-p do teste bilateral é:
    \[
    \text{valor-p} = 2 \times P(T_{15} \leq |-0.0026|) \approx 2 \times 0.499 \approx 0.998
    \]

    Como o valor-p é extremamente elevado (aproximadamente 0.998), muito superior a todos os níveis de significância usuais (1\%, 5\% e 10\%), não rejeitamos a hipótese nula $H_0: \beta_1 = 0$ para nenhum destes níveis.

    Isto indica que não há evidência estatística suficiente de uma relação linear significativa entre a altura das ondas e os estragos causados na orla costeira.

    \textbf{Resposta:} A opção correta é \textbf{(A) Não se rejeita $H_0 : \beta_1 = 0$ a 1\%, 5\% e 10\%}.
    \end{mdframed}

    \vspace{0.5cm}

    \item \textbf{A perda percentual de massa (\(Y\)) de uma certa substância metálica (quando exposta a oxigénio seco a 500°C) depende do período de exposição (\(x\), em horas). Um conjunto de 6 medições conduziram a:}

    \[
    \sum_{i=1}^6 x_i = 16,\quad 
    \sum_{i=1}^6 x_i^2 = 48.5,\quad 
    \sum_{i=1}^6 y_i = 0.207,\quad 
    \]

    \[
    \sum_{i=1}^6 y_i^2 = 0.007777,\quad 
    \sum_{i=1}^6 x_i y_i = 0.612
    \]

    \textbf{Preencha a caixa abaixo com o valor do coeficiente de determinação com, pelo menos, 4 casas decimais.}

    \vspace{0.3cm}

    \begin{mdframed}[backgroundcolor=gray!10,linewidth=0pt,innertopmargin=10pt,innerbottommargin=10pt]
    \textbf{Resolução:}

    Este problema envolve o cálculo do coeficiente de determinação $R^2$ numa regressão linear simples, que mede a proporção da variância de $Y$ explicada pelo modelo.

    Calculem-se primeiro as médias das variáveis:
    \[
    \bar{x} = \frac{16}{6} = 2.6667, \quad \bar{y} = \frac{0.207}{6} = 0.0345
    \]

    Em seguida, determinam-se os somatórios centrados necessários:
    \begin{align*}
    S_{xx} &= \sum x_i^2 - n\bar{x}^2 = 48.5 - 6 \times (2.6667)^2 = 5.8333 \\
    S_{yy} &= \sum y_i^2 - n\bar{y}^2 = 0.007777 - 6 \times (0.0345)^2 = 0.0006355 \\
    S_{xy} &= \sum x_i y_i - n\bar{x}\bar{y} = 0.612 - 6 \times 2.6667 \times 0.0345 = 0.0599931
    \end{align*}

    O coeficiente de determinação é dado pela fórmula:
    \[
    R^2 = \frac{(S_{xy})^2}{S_{xx} \times S_{yy}}
    \]

    Substituindo os valores calculados:
    \[
    R^2 = \frac{(0.0599931)^2}{5.8333 \times 0.0006355} = \frac{0.003599}{0.003706} \approx 0.9711
    \]

    Este resultado indica que aproximadamente 97.11% da variância na perda percentual de massa é explicada pelo período de exposição no modelo de regressão linear simples.

    \textbf{Resposta:} O coeficiente de determinação é 0.9711.
    \end{mdframed}

\end{enumerate}


\end{document}
