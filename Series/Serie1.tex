\documentclass[a4paper,12pt]{article}
\usepackage{titling}   
\usepackage[portuguese]{babel}
\usepackage{amsmath} 
\usepackage{graphicx}
\usepackage{xcolor}
\usepackage{titling}
\usepackage{amssymb}
\usepackage{tikz}
\usepackage{cancel}
\usepackage{mdframed}
\usetikzlibrary{automata, positioning}

\setlength{\droptitle}{-6em}
\pretitle{\begin{center}\LARGE}
\posttitle{\par\end{center}}
\preauthor{\begin{center}\large}
\postauthor{\end{center}} 
\predate{\begin{center}\large} 
\postdate{\end{center}} 

\author{Series de Problemas 1}

\begin{document}

\title{Probabilidade e Estatística}
\date{10 de Março de 2025}
\maketitle

\begin{enumerate}
  \item \textbf{Uma peça de certo tipo é classificada de acordo com a sua dimensão e porosidade. Num grande lote composto por peças deste tipo, verificaram-se as seguintes proporções: 44\% têm dimensão inadequada e são porosas; 53\% têm dimensão inadequada e não são porosas.
  Escolhida ao acaso uma peça do lote, calcule a probabilidade de ela ser porosa, sabendo que tem dimensão inadequada.
  Indique o resultado com pelo menos três casas decimais.}
  
  \vspace{0.3cm}
  
  \begin{mdframed}[backgroundcolor=gray!10, linewidth=0pt, innertopmargin=10pt, innerbottommargin=10pt]
  \textbf{Resolução:}
  
  Considere-se os eventos $D$ e $P$ como sendo, respectivamente, a peça ter dimensão inadequada e a peça ser porosa. Note-se que o enunciado fornece:
  \begin{align*}
    P(D \cap P) &= 0.44 \\
    P(D \cap \overline{P}) &= 0.53
  \end{align*}
  
  Pretende-se calcular $P(P|D)$. Observe-se que pelo Teorema da Probabilidade Condicional:
  \begin{align*}
     P(P|D) &= \frac{P(P \cap D)}{P(D)} \\
    &= \frac{P(P \cap D)}{P(P \cap D) + P(\overline{P} \cap D)} \\
    &= \displaystyle \frac{0.44}{0.44+0.53} = 0.4536...
  \end{align*}

  \textbf{Resposta:} A probabilidade da peça ser porosa, sabendo que tem dimensão inadequada, é 0.454.
  \end{mdframed}

  \vspace{0.5cm}
  
  \item \textbf{Sabe-se que a probabilidade de haver um dia de sol é 0.9. Sabe-se também que caso haja sol, a probabilidade de haver alunos na sala é de 0.22 e que caso não haja sol, a probabilidade de haver alunos na sala é 0.87 .
  Qual a probabilidade de haver alunos na sala?}
  \begin{enumerate}
    \item 0.1914
    \item 0.8766
    \item 0.285
    \item 0.87
  \end{enumerate}
  
  \vspace{0.3cm}
  
  \begin{mdframed}[backgroundcolor=gray!10, linewidth=0pt, innertopmargin=10pt, innerbottommargin=10pt]
  \textbf{Resolução:}
  
  Considere-se os eventos $S$ e $A$ como sendo, respectivamente, a ocorrência de sol e a presença de alunos na sala. O enunciado fornece:
  \begin{align*}
    P(S) &= 0.9 \\
    P(A|S) &= 0.22 \\
    P(A|\overline{S}) &= 0.87
  \end{align*}
  
  Para calcular a probabilidade de haver alunos na sala, aplique-se o Teorema da Probabilidade Total:
  \begin{align*}
    P(A) &= P(A|S)P(S) + P(A|\overline{S})P(\overline{S}) \\
    &= 0.22 \times 0.9 + 0.87 \times 0.1 = 0.285
  \end{align*}
  
  \textbf{Resposta:} A probabilidade de haver alunos na sala é 0.285, correspondendo à alternativa \textbf{(c)}.
  \end{mdframed}

  \vspace{0.5cm}
  
  \item \textbf{Num arranha-céus 50 \% das portas de segurança funcionam por controlo remoto. Durante um processo de inspeção regular, verificou-se que a probabilidade de uma porta não abrir é igual a 0.125 se esta opera remotamente, e é igual a 0.35 caso não opere remotamente.
  Considere que uma das portas foi selecionada ao acaso para inspeção. Qual é a probabilidade de a porta selecionada operar por controlo remoto, sabendo que ela abriu durante a inspeção?
  Indique o resultado com pelo menos quatro casas decimais.}
  
  \vspace{0.3cm}
  
  \begin{mdframed}[backgroundcolor=gray!10, linewidth=0pt, innertopmargin=10pt, innerbottommargin=10pt]
  \textbf{Resolução:}
  
  Defina-se os eventos $R$ e $A$ como sendo, respectivamente, a porta operar por controlo remoto e a porta abrir durante a inspeção. O enunciado fornece:
  \begin{align*}
    P(R) &= 0.5 \\
    P(\overline{R}) &= 0.5 \\
    P(\overline{A}|R) &= 0.125 \\
    P(\overline{A}|\overline{R}) &= 0.35
  \end{align*}
  
  Pretende-se calcular $P(R|A)$. Aplicando o Teorema de Bayes:
  \begin{align*}
    P(R|A) &= \frac{P(A|R)P(R)}{P(A)} \\
    &= \frac{P(A|R)P(R)}{P(A|R)P(R) + P(A|\overline{R})P(\overline{R})} \\
    &= \frac{(1 - P(\overline{A}|R))P(R)}{(1 - P(\overline{A}|R))P(R) + (1 - P(\overline{A}|\overline{R}))P(\overline{R})} \\
    &= \frac{(1 - 0.125) \times 0.5}{(1 - 0.125) \times 0.5 + (1 - 0.35) \times 0.5} = 0.57377...
  \end{align*}

  \textbf{Resposta:} A probabilidade da porta operar por controlo remoto, sabendo que abriu durante a inspeção, é 0.5738.
  \end{mdframed}

  \vspace{0.5cm}

  \item \textbf{Numa dada experiência aleatória, sejam $A$ e $B$ dois acontecimentos independentes, tais que $P(A)= \frac{1}{6}$ e $P(B)= \frac{1}{9}$ . Calcule $P[A|(A\cup B)]$. Preencha a caixa com o resultado com, pelo menos, duas casas decimais.}
  
  \vspace{0.3cm}
  
  \begin{mdframed}[backgroundcolor=gray!10, linewidth=0pt, innertopmargin=10pt, innerbottommargin=10pt]
  \textbf{Resolução:}
  
  Observe-se que, como $A$ e $B$ são independentes, tem-se $P(A \cap B) = P(A) \times P(B) = \frac{1}{6} \times \frac{1}{9} = \frac{1}{54}$. 
  
  Para calcular $P(A|(A\cup B))$, utilize-se a definição de probabilidade condicional:
  \begin{align*}
    P(A|(A\cup B)) &= \frac{P(A \cap (A\cup B))}{P(A\cup B)} \\
    &= \frac{P((A \cap A) \cup (A \cap B))}{P(A) + P(B) - P(A \cap B)} \\
    &= \frac{P(A\cap A) + P(A \cap B) - P(A \cap A \cap A \cap B)}{P(A) + P(B) - P(A \cap B)} \\
    &= \frac{P(A) + P(A \cap B) - P(A \cap B)}{P(A) + P(B) - P(A \cap B)} \\
    &= \frac{P(A)}{P(A) + P(B) - P(A \cap B)} \\
    &= \frac{\frac{1}{6}}{\frac{1}{6} + \frac{1}{9} - \frac{1}{54}} = \frac{9}{14} = 0.642857...
  \end{align*}

  \textbf{Resposta:} A probabilidade $P[A|(A\cup B)]$ é 0.64.
  \end{mdframed}

  \vspace{0.5cm}

  \item \textbf{Considere os acontecimentos $A$, $B$ e $C$ tais que: $A$ e $B$ são independentes condicionalmente a $C$; $P(A|C)= \frac{1}{80}$ , $P(A \cap B)= \frac{9}{32}$ e $P(B \cap C)= \frac{3}{4}$ . Obtenha $P(C|(A \cap B))$, indicando o resultado com pelo menos quatro casas decimais.}
  
  \vspace{0.3cm}
  
  \begin{mdframed}[backgroundcolor=gray!10, linewidth=0pt, innertopmargin=10pt, innerbottommargin=10pt]
  \textbf{Resolução:}
  
  Note-se que, como $A$ e $B$ são independentes condicionalmente a $C$, temos $P(A \cap B|C) = P(A|C) \times P(B|C)$. 
  
  Aplique-se o Teorema de Bayes para calcular $P(C|(A \cap B))$:
  \begin{align*}
    P(C|(A \cap B)) &= \frac{P((A \cap B) | C) P(C)}{P(A \cap B)} \\
    &= \frac{P(A|C)P(B|C)P(C)}{P(A \cap B)} \\
  \end{align*}
  
  Observe-se que $P(B|C) = \frac{P(B \cap C)}{P(C)}$, logo $P(B|C)P(C) = P(B \cap C)$. Continuando:
  \begin{align*}
    P(C|(A \cap B)) &= \frac{P(A|C)P(B \cap C)}{P(A \cap B)} \\
    &= \frac{\frac{1}{80} \times \frac{3}{4}}{\frac{9}{32}} = \frac{1}{30} = 0.033333...
  \end{align*}

  \textbf{Resposta:} A probabilidade $P(C|(A \cap B))$ é 0.0333.
  \end{mdframed}
\end{enumerate}
\end{document}