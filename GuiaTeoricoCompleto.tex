\documentclass[a4paper,12pt]{article}
\usepackage[portuguese]{babel}
\usepackage[utf8]{inputenc}
\usepackage{amsmath}
\usepackage{amssymb}
\usepackage{graphicx}
\usepackage{tikz}
\usepackage{pgfplots}
\usepackage{float}
\usepackage{xcolor}
\usepackage{booktabs}
\usepackage{multicol}
\usepackage{tcolorbox}
\usepackage{enumitem}
\usepackage{mdframed}
\usetikzlibrary{automata, positioning, arrows, shapes.geometric}

\pgfplotsset{compat=1.18}

\setlength{\parindent}{0pt}
\setlength{\parskip}{0.5em}

\title{\Huge \textbf{Guia Teórico Completo de Probabilidade e Estatística}}
\author{Vicente Duarte}
\date{Ano Letivo 2024/2025}

\begin{document}

\maketitle
\thispagestyle{empty}

\begin{center}
\rule{\textwidth}{1pt}
\end{center}

\begin{abstract}
Este guia teórico apresenta uma visão abrangente dos conceitos de Probabilidade e Estatística abordados nas séries de problemas do curso. O documento está estruturado de forma progressiva, abordando desde conceitos básicos de probabilidade até distribuições de variáveis aleatórias, probabilidade condicional e técnicas de inferência estatística. É especialmente útil como material de apoio ao estudo e à resolução dos exercícios propostos.
\end{abstract}

\tableofcontents
\newpage

\section{Conceitos Fundamentais de Probabilidade}

\subsection{Espaço Amostral e Eventos}

O espaço amostral, geralmente representado por $\Omega$, é o conjunto de todos os resultados possíveis de uma experiência aleatória. Um evento $A$ é um subconjunto do espaço amostral, ou seja, $A \subseteq \Omega$.

\begin{tcolorbox}[colback=gray!10, colframe=black, title=\textbf{Exemplo}]
Ao lançar um dado de seis faces, o espaço amostral é $\Omega = \{1, 2, 3, 4, 5, 6\}$. O evento "sair um número par" corresponde ao conjunto $A = \{2, 4, 6\}$.
\end{tcolorbox}

\subsection{Definição de Probabilidade}

A probabilidade de um evento $A$, denotada por $P(A)$, é uma medida da possibilidade de $A$ ocorrer. Pela definição axiomática de Kolmogorov, a probabilidade deve satisfazer:

\begin{enumerate}[label=(\roman*)]
    \item $P(A) \geq 0$ para qualquer evento $A$
    \item $P(\Omega) = 1$
    \item Se $A_1, A_2, \ldots$ são eventos mutuamente exclusivos, então $P(A_1 \cup A_2 \cup \ldots) = P(A_1) + P(A_2) + \ldots$
\end{enumerate}

\begin{tcolorbox}[colback=gray!10, colframe=black, title=\textbf{Exemplo}]
No lançamento de uma moeda equilibrada, $P(\text{cara}) = P(\text{coroa}) = \frac{1}{2}$.

Para o lançamento de um dado equilibrado, $P(\text{sair 3}) = \frac{1}{6}$.
\end{tcolorbox}

\subsection{Operações com Eventos}

As operações básicas com eventos incluem:

\begin{itemize}
    \item \textbf{União} ($A \cup B$): evento que ocorre quando $A$ ou $B$ (ou ambos) ocorrem
    \item \textbf{Interseção} ($A \cap B$): evento que ocorre quando $A$ e $B$ ocorrem simultaneamente
    \item \textbf{Complemento} ($\overline{A}$ ou $A^c$): evento que ocorre quando $A$ não ocorre
\end{itemize}

\begin{figure}[H]
\centering
\begin{tikzpicture}
    % Espaço amostral (retângulo)
    \draw[thick] (0,0) rectangle (6,4);
    \node at (0.5,3.5) {$\Omega$};
    
    % Conjunto A (círculo à esquerda)
    \draw[thick, fill=blue!20] (2,2) circle (1.2);
    \node at (1.3,2) {$A$};
    
    % Conjunto B (círculo à direita)
    \draw[thick, fill=red!20] (4,2) circle (1.2);
    \node at (4.7,2) {$B$};
    
    % Interseção A e B (área sobreposta)
    \begin{scope}
        \clip (2,2) circle (1.2);
        \fill[green!20] (4,2) circle (1.2);
    \end{scope}
    \node at (3,2) {$A \cap B$};
\end{tikzpicture}
\caption{Diagrama de Venn representando eventos $A$ e $B$, sua interseção e o espaço amostral $\Omega$}
\end{figure}

\subsection{Probabilidade Condicional}

A probabilidade condicional de um evento $A$ dado que o evento $B$ ocorreu, denotada por $P(A|B)$, é definida como:

\begin{equation}
P(A|B) = \frac{P(A \cap B)}{P(B)}, \quad \text{onde } P(B) > 0
\end{equation}

\begin{tcolorbox}[colback=blue!5, colframe=blue!40, title=\textbf{Fórmula Fundamental}]
A fórmula da probabilidade condicional pode ser reorganizada para obter:
\begin{equation}
P(A \cap B) = P(B) \times P(A|B) = P(A) \times P(B|A)
\end{equation}
Esta é uma das fórmulas mais importantes e frequentemente utilizadas em problemas de probabilidade.
\end{tcolorbox}

\subsection{Independência de Eventos}

Dois eventos $A$ e $B$ são independentes se e somente se:

\begin{equation}
P(A \cap B) = P(A) \times P(B)
\end{equation}

Equivalentemente, $A$ e $B$ são independentes se $P(A|B) = P(A)$ ou $P(B|A) = P(B)$.

\begin{tcolorbox}[colback=gray!10, colframe=black, title=\textbf{Exemplo}]
Ao lançar um dado duas vezes, o evento "tirar um número par no primeiro lançamento" é independente do evento "tirar um número menor que 3 no segundo lançamento".
\end{tcolorbox}

\subsection{Teorema da Probabilidade Total}

Se $B_1, B_2, \ldots, B_n$ constituem uma partição do espaço amostral $\Omega$ (ou seja, são mutuamente exclusivos e $B_1 \cup B_2 \cup \ldots \cup B_n = \Omega$), então para qualquer evento $A$:

\begin{equation}
P(A) = \sum_{i=1}^{n} P(A|B_i) \times P(B_i)
\end{equation}

\begin{figure}[H]
\centering
\begin{tikzpicture}
    % Espaço amostral (retângulo)
    \draw[thick] (0,0) rectangle (8,4);
    \node at (0.5,3.5) {$\Omega$};
    
    % Partição do espaço amostral
    \fill[blue!10] (0,0) rectangle (2.66,4);
    \fill[green!10] (2.66,0) rectangle (5.33,4);
    \fill[red!10] (5.33,0) rectangle (8,4);
    
    \node at (1.33,2) {$B_1$};
    \node at (4,2) {$B_2$};
    \node at (6.66,2) {$B_3$};
    
    % Evento A (região com contorno pontilhado)
    \draw[thick, dashed] (1,1) .. controls (2,3) and (4,3.5) .. (7,2) .. controls (6,0.5) and (3,0.5) .. (1,1);
    \node at (3.5,1.5) {$A$};
\end{tikzpicture}
\caption{Ilustração do Teorema da Probabilidade Total: evento $A$ distribuído entre a partição $\{B_1, B_2, B_3\}$}
\end{figure}

\subsection{Teorema de Bayes}

O Teorema de Bayes permite inverter condições em probabilidades condicionais:

\begin{equation}
P(B_i|A) = \frac{P(A|B_i) \times P(B_i)}{\sum_{j=1}^{n} P(A|B_j) \times P(B_j)}
\end{equation}

Numa forma mais simplificada, para dois eventos $A$ e $B$:

\begin{equation}
P(B|A) = \frac{P(A|B) \times P(B)}{P(A)}
\end{equation}

\begin{tcolorbox}[colback=orange!5, colframe=orange!50, title=\textbf{Dica para Aplicar o Teorema de Bayes}]
O Teorema de Bayes é particularmente útil em problemas onde:
\begin{itemize}
    \item São fornecidas probabilidades do tipo $P(A|B)$
    \item Pede-se para calcular probabilidades do tipo $P(B|A)$
    \item Existe informação adicional que surge após uma observação inicial
\end{itemize}
\end{tcolorbox}

\section{Variáveis Aleatórias}

\subsection{Definição e Tipos}

Uma variável aleatória é uma função que associa um valor numérico a cada resultado do espaço amostral. As variáveis aleatórias podem ser:

\begin{itemize}
    \item \textbf{Discretas}: assumem valores em conjunto finito ou enumerável (e.g., número de caras em 10 lançamentos de moeda)
    \item \textbf{Contínuas}: podem assumir qualquer valor em um intervalo contínuo (e.g., tempo de espera, altura de uma pessoa)
\end{itemize}

\subsection{Função de Probabilidade (V.A. Discretas)}

Para uma variável aleatória discreta $X$, a função de probabilidade $p_X(x)$ (também chamada de função massa de probabilidade) é definida como:

\begin{equation}
p_X(x) = P(X = x)
\end{equation}

Propriedades:
\begin{enumerate}[label=(\roman*)]
    \item $p_X(x) \geq 0$ para todo $x$
    \item $\sum_{x} p_X(x) = 1$
\end{enumerate}

\subsection{Função Densidade de Probabilidade (V.A. Contínuas)}

Para uma variável aleatória contínua $X$, a função densidade de probabilidade $f_X(x)$ satisfaz:

\begin{equation}
P(a \leq X \leq b) = \int_{a}^{b} f_X(x) \, dx
\end{equation}

Propriedades:
\begin{enumerate}[label=(\roman*)]
    \item $f_X(x) \geq 0$ para todo $x$
    \item $\int_{-\infty}^{\infty} f_X(x) \, dx = 1$
\end{enumerate}

\subsection{Função de Distribuição Acumulada}

A função de distribuição acumulada (FDA) $F_X(x)$ de uma variável aleatória $X$ é definida como:

\begin{equation}
F_X(x) = P(X \leq x)
\end{equation}

Para variáveis aleatórias discretas:
\begin{equation}
F_X(x) = \sum_{t \leq x} p_X(t)
\end{equation}

Para variáveis aleatórias contínuas:
\begin{equation}
F_X(x) = \int_{-\infty}^{x} f_X(t) \, dt
\end{equation}

\begin{figure}[H]
\centering
\begin{tikzpicture}
    \begin{axis}[
        width=10cm,
        height=6cm,
        axis lines=middle,
        xlabel=$x$,
        ylabel=$F_X(x)$,
        xtick={-2, -1, 0, 1, 2, 3},
        ytick={0, 0.25, 0.5, 0.75, 1},
        xmin=-2.5,
        xmax=3.5,
        ymin=-0.1,
        ymax=1.1,
        legend pos=north west,
    ]
    
    % Função de distribuição contínua (exemplo: normal padrão)
    \addplot[thick, blue, domain=-2.5:3.5, samples=100] {0.5*(1+tanh(0.5*x*sqrt(2/pi)))};
    \addlegendentry{$F_X(x)$ contínua};
    
    % Função de distribuição discreta (exemplo: binomial)
    \addplot[thick, red, mark=*, only marks] coordinates {
        (-2, 0)
        (-1, 0.125)
        (0, 0.5)
        (1, 0.875)
        (2, 1)
    };
    
    \addplot[thick, red, const plot] coordinates {
        (-2, 0)
        (-1, 0)
        (-1, 0.125)
        (0, 0.125)
        (0, 0.5)
        (1, 0.5)
        (1, 0.875)
        (2, 0.875)
        (2, 1)
        (3, 1)
    };
    \addlegendentry{$F_X(x)$ discreta};
    
    \end{axis}
\end{tikzpicture}
\caption{Comparação entre funções de distribuição acumulada para variáveis aleatórias contínuas e discretas}
\end{figure}

\subsection{Valor Esperado (Média)}

O valor esperado de uma variável aleatória $X$, denotado por $E[X]$ ou $\mu_X$, representa o valor médio que esperamos obter se repetirmos a experiência muitas vezes.

Para variáveis aleatórias discretas:
\begin{equation}
E[X] = \sum_{x} x \cdot p_X(x)
\end{equation}

Para variáveis aleatórias contínuas:
\begin{equation}
E[X] = \int_{-\infty}^{\infty} x \cdot f_X(x) \, dx
\end{equation}

\subsection{Variância e Desvio Padrão}

A variância de uma variável aleatória $X$, denotada por $\text{Var}(X)$ ou $\sigma_X^2$, mede a dispersão dos valores em torno da média:

\begin{equation}
\text{Var}(X) = E[(X - \mu_X)^2] = E[X^2] - (E[X])^2
\end{equation}

O desvio padrão $\sigma_X$ é definido como a raiz quadrada da variância:

\begin{equation}
\sigma_X = \sqrt{\text{Var}(X)}
\end{equation}

\begin{tcolorbox}[colback=blue!5, colframe=blue!40, title=\textbf{Propriedades do Valor Esperado e Variância}]
Para constantes $a$ e $b$ e variáveis aleatórias $X$ e $Y$:

\begin{align}
E[aX + b] &= aE[X] + b \\
\text{Var}(aX + b) &= a^2\text{Var}(X) \\
\end{align}

Se $X$ e $Y$ são independentes:
\begin{align}
E[X \cdot Y] &= E[X] \cdot E[Y] \\
\text{Var}(X + Y) &= \text{Var}(X) + \text{Var}(Y)
\end{align}
\end{tcolorbox}

\section{Distribuições Discretas Importantes}

\subsection{Distribuição de Bernoulli}

A distribuição de Bernoulli modela experiências com apenas dois resultados possíveis (sucesso ou fracasso). Se $X$ é uma variável aleatória de Bernoulli com parâmetro $p$ (probabilidade de sucesso), então:

\begin{equation}
P(X = x) = 
\begin{cases}
p, & \text{se } x = 1 \\
1-p, & \text{se } x = 0
\end{cases}
\end{equation}

Média: $E[X] = p$

Variância: $\text{Var}(X) = p(1-p)$

\begin{figure}[H]
\centering
\begin{tikzpicture}
    \begin{axis}[
        width=8cm,
        height=5cm,
        ybar,
        enlargelimits=0.15,
        ylabel={$P(X=x)$},
        xlabel={$x$},
        ymin=0,
        symbolic x coords={0,1},
        xtick=data,
        nodes near coords,
        legend pos=north west,
        ]
        \addplot coordinates {(0,0.3) (1,0.7)};
        \addlegendentry{$p=0.7$}
        
        \addplot coordinates {(0,0.5) (1,0.5)};
        \addlegendentry{$p=0.5$}
        
        \addplot coordinates {(0,0.8) (1,0.2)};
        \addlegendentry{$p=0.2$}
    \end{axis}
\end{tikzpicture}
\caption{Função de massa de probabilidade da distribuição de Bernoulli para diferentes valores de $p$}
\end{figure}

\subsection{Distribuição Binomial}

A distribuição binomial modela o número de sucessos em $n$ tentativas independentes de Bernoulli, cada uma com probabilidade de sucesso $p$. Se $X$ é uma variável aleatória binomial com parâmetros $n$ e $p$, denotamos $X \sim \text{Bin}(n, p)$ e:

\begin{equation}
P(X = k) = \binom{n}{k} p^k (1-p)^{n-k}, \quad k = 0, 1, 2, \ldots, n
\end{equation}

onde $\binom{n}{k} = \frac{n!}{k!(n-k)!}$

Média: $E[X] = np$

Variância: $\text{Var}(X) = np(1-p)$

\begin{figure}[H]
\centering
\begin{tikzpicture}
    \begin{axis}[
        width=10cm,
        height=6cm,
        ybar,
        bar width=5pt,
        enlargelimits=0.15,
        ylabel={$P(X=k)$},
        xlabel={$k$},
        ymin=0,
        xtick={0,1,2,3,4,5,6,7,8,9,10},
        legend pos=north east,
        ]
        \addplot coordinates {
            (0,0.0010) (1,0.0098) (2,0.0439) (3,0.1172) (4,0.2051) (5,0.2461) 
            (6,0.2051) (7,0.1172) (8,0.0439) (9,0.0098) (10,0.0010)
        };
        \addlegendentry{$n=10, p=0.5$}
        
        \addplot coordinates {
            (0,0.3487) (1,0.3874) (2,0.1937) (3,0.0574) (4,0.0112) (5,0.0015)
            (6,0.0001) (7,0.0000) (8,0.0000) (9,0.0000) (10,0.0000)
        };
        \addlegendentry{$n=10, p=0.1$}
    \end{axis}
\end{tikzpicture}
\caption{Função de massa de probabilidade da distribuição binomial para diferentes valores de $p$ com $n=10$}
\end{figure}

\subsection{Distribuição Geométrica}

A distribuição geométrica modela o número de tentativas necessárias até o primeiro sucesso em tentativas independentes de Bernoulli. Se $X$ é uma variável aleatória geométrica com parâmetro $p$ (probabilidade de sucesso em cada tentativa), então:

\begin{equation}
P(X = k) = (1-p)^{k-1} p, \quad k = 1, 2, 3, \ldots
\end{equation}

Média: $E[X] = \frac{1}{p}$

Variância: $\text{Var}(X) = \frac{1-p}{p^2}$

\begin{tcolorbox}[colback=green!5, colframe=green!40, title=\textbf{Propriedade de "Falta de Memória"}]
A distribuição geométrica é a única distribuição discreta que possui a propriedade de "falta de memória":
\begin{equation}
P(X > n+m | X > n) = P(X > m)
\end{equation}
Esta propriedade significa que, se já esperamos $n$ tentativas sem sucesso, a probabilidade de precisarmos de mais $m$ tentativas é igual à probabilidade original de precisarmos de $m$ tentativas.
\end{tcolorbox}

\subsection{Distribuição de Poisson}

A distribuição de Poisson modela o número de ocorrências de um evento em um intervalo fixo, quando esses eventos ocorrem a uma taxa média constante e independentemente uns dos outros. Se $X$ é uma variável aleatória de Poisson com parâmetro $\lambda$ (taxa média de ocorrências), denotamos $X \sim \text{Poisson}(\lambda)$ e:

\begin{equation}
P(X = k) = \frac{e^{-\lambda} \lambda^k}{k!}, \quad k = 0, 1, 2, \ldots
\end{equation}

Média: $E[X] = \lambda$

Variância: $\text{Var}(X) = \lambda$

\begin{figure}[H]
\centering
\begin{tikzpicture}
    \begin{axis}[
        width=10cm,
        height=6cm,
        ybar,
        bar width=4pt,
        enlargelimits=0.15,
        ylabel={$P(X=k)$},
        xlabel={$k$},
        ymin=0,
        xtick={0,1,2,3,4,5,6,7,8,9,10,11,12,13,14,15},
        legend pos=north east,
        ]
        \addplot coordinates {
            (0,0.0498) (1,0.1494) (2,0.2240) (3,0.2240) (4,0.1680) (5,0.1008) 
            (6,0.0504) (7,0.0216) (8,0.0081) (9,0.0027) (10,0.0008) (11,0.0002)
            (12,0.0001) (13,0.0000) (14,0.0000) (15,0.0000)
        };
        \addlegendentry{$\lambda=3$}
        
        \addplot coordinates {
            (0,0.0067) (1,0.0337) (2,0.0842) (3,0.1404) (4,0.1755) (5,0.1755) 
            (6,0.1462) (7,0.1044) (8,0.0653) (9,0.0363) (10,0.0181) (11,0.0082)
            (12,0.0034) (13,0.0013) (14,0.0005) (15,0.0002)
        };
        \addlegendentry{$\lambda=5$}
    \end{axis}
\end{tikzpicture}
\caption{Função de massa de probabilidade da distribuição de Poisson para diferentes valores de $\lambda$}
\end{figure}

\section{Distribuições Contínuas Importantes}

\subsection{Distribuição Uniforme Contínua}

A distribuição uniforme contínua modela situações onde todos os valores em um intervalo $[a,b]$ têm a mesma probabilidade de ocorrer. Se $X$ é uma variável aleatória uniforme no intervalo $[a,b]$, denotamos $X \sim \text{Uniforme}(a,b)$ e sua função densidade de probabilidade é:

\begin{equation}
f_X(x) = 
\begin{cases}
\frac{1}{b-a}, & \text{se } a \leq x \leq b \\
0, & \text{caso contrário}
\end{cases}
\end{equation}

Função de distribuição acumulada:
\begin{equation}
F_X(x) = 
\begin{cases}
0, & \text{se } x < a \\
\frac{x-a}{b-a}, & \text{se } a \leq x \leq b \\
1, & \text{se } x > b
\end{cases}
\end{equation}

Média: $E[X] = \frac{a+b}{2}$

Variância: $\text{Var}(X) = \frac{(b-a)^2}{12}$

\begin{figure}[H]
\centering
\begin{tikzpicture}
    \begin{axis}[
        width=5cm,
        height=4cm,
        axis lines=middle,
        xlabel=$x$,
        ylabel=$f_X(x)$,
        xmin=-1,
        xmax=6,
        ymin=-0.1,
        ymax=0.6,
        title={Função Densidade},
    ]
    
    % Uniforme(0,5)
    \addplot[thick, domain=-1:6, samples=100, blue] coordinates {
        (-1, 0) (0, 0) (0, 0.2) (5, 0.2) (5, 0) (6, 0)
    };
    
    \end{axis}
\end{tikzpicture}
\hfill
\begin{tikzpicture}
    \begin{axis}[
        width=5cm,
        height=4cm,
        axis lines=middle,
        xlabel=$x$,
        ylabel=$F_X(x)$,
        xmin=-1,
        xmax=6,
        ymin=-0.1,
        ymax=1.1,
        title={Função de Distribuição},
    ]
    
    % Uniforme(0,5)
    \addplot[thick, domain=-1:6, samples=100, blue] coordinates {
        (-1, 0) (0, 0) (5, 1) (6, 1)
    };
    
    \end{axis}
\end{tikzpicture}
\caption{Funções de densidade e de distribuição da distribuição uniforme contínua no intervalo $[0,5]$}
\end{figure}

\subsection{Distribuição Exponencial}

A distribuição exponencial é frequentemente usada para modelar o tempo até a ocorrência de um evento, quando a taxa de ocorrência é constante. Se $X$ é uma variável aleatória exponencial com parâmetro $\lambda > 0$ (taxa), denotamos $X \sim \text{Exp}(\lambda)$ e sua função densidade de probabilidade é:

\begin{equation}
f_X(x) = 
\begin{cases}
\lambda e^{-\lambda x}, & \text{se } x \geq 0 \\
0, & \text{se } x < 0
\end{cases}
\end{equation}

Função de distribuição acumulada:
\begin{equation}
F_X(x) = 
\begin{cases}
1 - e^{-\lambda x}, & \text{se } x \geq 0 \\
0, & \text{se } x < 0
\end{cases}
\end{equation}

Média: $E[X] = \frac{1}{\lambda}$

Variância: $\text{Var}(X) = \frac{1}{\lambda^2}$

\begin{tcolorbox}[colback=green!5, colframe=green!40, title=\textbf{Propriedade de "Falta de Memória"}]
Assim como a distribuição geométrica, a distribuição exponencial possui a propriedade de "falta de memória":
\begin{equation}
P(X > s+t | X > t) = P(X > s)
\end{equation}
Isso significa que, se um componente já funcionou por $t$ unidades de tempo sem falhar, a probabilidade de ele funcionar por mais $s$ unidades de tempo é igual à probabilidade original de funcionar por $s$ unidades de tempo.
\end{tcolorbox}

\begin{figure}[H]
\centering
\begin{tikzpicture}
    \begin{axis}[
        width=10cm,
        height=6cm,
        domain=0:5,
        samples=100,
        axis lines=middle,
        xlabel=$x$,
        ylabel=$f_X(x)$,
        xmin=-0.5,
        xmax=5,
        ymin=-0.1,
        ymax=1.1,
        legend pos=north east,
        ]
        \addplot[thick, blue, domain=0:5] {1*exp(-1*x)};
        \addlegendentry{$\lambda=1$}
        
        \addplot[thick, red, domain=0:5] {0.5*exp(-0.5*x)};
        \addlegendentry{$\lambda=0.5$}
        
        \addplot[thick, green, domain=0:5] {2*exp(-2*x)};
        \addlegendentry{$\lambda=2$}
    \end{axis}
\end{tikzpicture}
\caption{Função densidade de probabilidade da distribuição exponencial para diferentes valores de $\lambda$}
\end{figure}

\subsection{Distribuição Normal}

A distribuição normal (ou gaussiana) é uma das distribuições mais importantes em estatística. Se $X$ é uma variável aleatória normal com média $\mu$ e variância $\sigma^2$, denotamos $X \sim N(\mu, \sigma^2)$ e sua função densidade de probabilidade é:

\begin{equation}
f_X(x) = \frac{1}{\sigma\sqrt{2\pi}} e^{-\frac{(x-\mu)^2}{2\sigma^2}}, \quad -\infty < x < \infty
\end{equation}

A distribuição normal padrão, denotada por $Z \sim N(0, 1)$, tem média zero e variância unitária. Qualquer variável aleatória normal pode ser padronizada através da transformação $Z = \frac{X - \mu}{\sigma}$.

Média: $E[X] = \mu$

Variância: $\text{Var}(X) = \sigma^2$

\begin{figure}[H]
\centering
\begin{tikzpicture}
    \begin{axis}[
        width=10cm,
        height=6cm,
        domain=-4:4,
        samples=100,
        axis lines=middle,
        xlabel=$x$,
        ylabel=$f_X(x)$,
        xmin=-4.5,
        xmax=4.5,
        ymin=-0.05,
        ymax=0.45,
        legend pos=north east,
        ]
        \addplot[thick, blue, domain=-4:4] {1/(1*sqrt(2*pi))*exp(-(x-0)^2/(2*1^2))};
        \addlegendentry{$\mu=0, \sigma=1$}
        
        \addplot[thick, red, domain=-4:4] {1/(1.5*sqrt(2*pi))*exp(-(x-0)^2/(2*1.5^2))};
        \addlegendentry{$\mu=0, \sigma=1.5$}
        
        \addplot[thick, green, domain=-4:4] {1/(1*sqrt(2*pi))*exp(-(x-2)^2/(2*1^2))};
        \addlegendentry{$\mu=2, \sigma=1$}
    \end{axis}
\end{tikzpicture}
\caption{Função densidade de probabilidade da distribuição normal para diferentes valores de $\mu$ e $\sigma$}
\end{figure}

\begin{tcolorbox}[colback=yellow!5, colframe=yellow!40, title=\textbf{Propriedades da Distribuição Normal}]
\begin{itemize}
    \item É simétrica em torno da média $\mu$
    \item A moda e a mediana coincidem com a média
    \item Tem forma de sino, com ponto de inflexão em $\mu \pm \sigma$
    \item Aproximadamente 68,27\% dos valores estão a uma distância de até $\sigma$ da média
    \item Aproximadamente 95,45\% dos valores estão a uma distância de até $2\sigma$ da média
    \item Aproximadamente 99,73\% dos valores estão a uma distância de até $3\sigma$ da média (regra 3-sigma)
\end{itemize}
\end{tcolorbox}

\begin{figure}[H]
\centering
\begin{tikzpicture}
    \begin{axis}[
        width=10cm,
        height=6cm,
        domain=-4:4,
        samples=100,
        axis lines=middle,
        xlabel=$x$,
        ylabel=$f_X(x)$,
        xmin=-4,
        xmax=4,
        ymin=0,
        ymax=0.45,
        legend pos=north east,
        ]
        \addplot[thick, blue, domain=-4:4] {1/(1*sqrt(2*pi))*exp(-(x-0)^2/(2*1^2))};
        
        \addplot[fill=blue!20, domain=-1:1, samples=100] {1/(1*sqrt(2*pi))*exp(-(x-0)^2/(2*1^2))} \closedcycle;
        \node at (axis cs:0,0.1) {68.27\%};
        
        \addplot[fill=blue!10, domain=-2:-1, samples=100] {1/(1*sqrt(2*pi))*exp(-(x-0)^2/(2*1^2))} \closedcycle;
        \addplot[fill=blue!10, domain=1:2, samples=100] {1/(1*sqrt(2*pi))*exp(-(x-0)^2/(2*1^2))} \closedcycle;
        \node at (axis cs:-1.5,0.05) {13.59\%};
        \node at (axis cs:1.5,0.05) {13.59\%};
        
        \addplot[fill=blue!5, domain=-3:-2, samples=100] {1/(1*sqrt(2*pi))*exp(-(x-0)^2/(2*1^2))} \closedcycle;
        \addplot[fill=blue!5, domain=2:3, samples=100] {1/(1*sqrt(2*pi))*exp(-(x-0)^2/(2*1^2))} \closedcycle;
        \node at (axis cs:-2.5,0.02) {2.14\%};
        \node at (axis cs:2.5,0.02) {2.14\%};
    \end{axis}
\end{tikzpicture}
\caption{Distribuição normal padrão com áreas correspondentes a diferentes intervalos de desvios padrão}
\end{figure}

\section{Variáveis Aleatórias Multidimensionais}

\subsection{Distribuição Conjunta}

Para variáveis aleatórias discretas $X$ e $Y$, a função de probabilidade conjunta é:

\begin{equation}
p_{X,Y}(x,y) = P(X = x, Y = y)
\end{equation}

Para variáveis aleatórias contínuas $X$ e $Y$, a função densidade de probabilidade conjunta satisfaz:

\begin{equation}
P((X,Y) \in A) = \iint_A f_{X,Y}(x,y) \, dx \, dy
\end{equation}

\subsection{Distribuições Marginais}

Para variáveis aleatórias discretas:
\begin{align}
p_X(x) &= \sum_y p_{X,Y}(x,y) \\
p_Y(y) &= \sum_x p_{X,Y}(x,y)
\end{align}

Para variáveis aleatórias contínuas:
\begin{align}
f_X(x) &= \int_{-\infty}^{\infty} f_{X,Y}(x,y) \, dy \\
f_Y(y) &= \int_{-\infty}^{\infty} f_{X,Y}(x,y) \, dx
\end{align}

\subsection{Distribuições Condicionais}

Para variáveis aleatórias discretas:
\begin{align}
p_{X|Y}(x|y) &= \frac{p_{X,Y}(x,y)}{p_Y(y)}, \quad \text{se } p_Y(y) > 0 \\
p_{Y|X}(y|x) &= \frac{p_{X,Y}(x,y)}{p_X(x)}, \quad \text{se } p_X(x) > 0
\end{align}

Para variáveis aleatórias contínuas:
\begin{align}
f_{X|Y}(x|y) &= \frac{f_{X,Y}(x,y)}{f_Y(y)}, \quad \text{se } f_Y(y) > 0 \\
f_{Y|X}(y|x) &= \frac{f_{X,Y}(x,y)}{f_X(x)}, \quad \text{se } f_X(x) > 0
\end{align}

\subsection{Independência de Variáveis Aleatórias}

Duas variáveis aleatórias $X$ e $Y$ são independentes se e somente se:

Para variáveis aleatórias discretas:
\begin{equation}
p_{X,Y}(x,y) = p_X(x) \cdot p_Y(y), \quad \text{para todo } x, y
\end{equation}

Para variáveis aleatórias contínuas:
\begin{equation}
f_{X,Y}(x,y) = f_X(x) \cdot f_Y(y), \quad \text{para todo } x, y
\end{equation}

\subsection{Covariância e Correlação}

A covariância entre duas variáveis aleatórias $X$ e $Y$ é definida como:

\begin{equation}
\text{Cov}(X,Y) = E[(X - \mu_X)(Y - \mu_Y)] = E[XY] - E[X]E[Y]
\end{equation}

onde $\mu_X = E[X]$ e $\mu_Y = E[Y]$.

A correlação entre $X$ e $Y$ é:

\begin{equation}
\rho_{X,Y} = \frac{\text{Cov}(X,Y)}{\sigma_X \sigma_Y}
\end{equation}

onde $\sigma_X = \sqrt{\text{Var}(X)}$ e $\sigma_Y = \sqrt{\text{Var}(Y)}$.

\begin{tcolorbox}[colback=blue!5, colframe=blue!40, title=\textbf{Propriedades da Correlação}]
\begin{itemize}
    \item $-1 \leq \rho_{X,Y} \leq 1$
    \item $\rho_{X,Y} = 1$ ou $\rho_{X,Y} = -1$ se e somente se existe uma relação linear perfeita entre $X$ e $Y$
    \item $\rho_{X,Y} = 0$ se $X$ e $Y$ são independentes (a recíproca não é necessariamente verdadeira)
\end{itemize}
\end{tcolorbox}

\subsection{Esperança Condicional}

A esperança condicional de uma variável aleatória $X$ dado $Y = y$ é definida como:

Para variáveis aleatórias discretas:
\begin{equation}
E[X|Y=y] = \sum_x x \cdot p_{X|Y}(x|y)
\end{equation}

Para variáveis aleatórias contínuas:
\begin{equation}
E[X|Y=y] = \int_{-\infty}^{\infty} x \cdot f_{X|Y}(x|y) \, dx
\end{equation}

A esperança condicional $E[X|Y]$ pode ser vista como uma função de $Y$.

\begin{tcolorbox}[colback=green!5, colframe=green!40, title=\textbf{Lei da Esperança Total}]
\begin{equation}
E[X] = E[E[X|Y]]
\end{equation}

Ou seja, a esperança incondicional de $X$ pode ser obtida tomando a média ponderada das esperanças condicionais.
\end{tcolorbox}

\section{Teoremas Limite e Aproximações}

\subsection{Lei dos Grandes Números}

A Lei dos Grandes Números assegura que a média aritmética de um grande número de variáveis aleatórias independentes e identicamente distribuídas converge para a esperança comum dessas variáveis.

Seja $X_1, X_2, \ldots, X_n$ uma sequência de variáveis aleatórias independentes e identicamente distribuídas com $E[X_i] = \mu$. Defina $\overline{X}_n = \frac{1}{n}(X_1 + X_2 + \ldots + X_n)$. Então:

\begin{equation}
\overline{X}_n \xrightarrow{P} \mu \quad \text{quando } n \to \infty
\end{equation}

onde $\xrightarrow{P}$ denota convergência em probabilidade.

\begin{figure}[H]
\centering
\begin{tikzpicture}
\begin{axis}[
    width=10cm,
    height=6cm,
    xlabel={$n$ (número de lançamentos)},
    ylabel={Frequência relativa de caras},
    xmin=0,
    xmax=100,
    ymin=0,
    ymax=1,
    grid=major,
    legend pos=south east,
    ]
    \addplot[thick, blue, domain=1:100, samples=100] {0.5 + 0.5*sin(deg(x))/sqrt(x)};
    \addplot[thick, red, domain=1:100, samples=100] {0.5 + 0.25*cos(deg(2*x))/sqrt(x)};
    \addplot[thick, green, domain=1:100, samples=100] {0.5 - 0.4*sin(deg(0.5*x))/sqrt(x)};
    \addplot[thick, black, domain=1:100, samples=100] {0.5};
    \legend{Simulação 1, Simulação 2, Simulação 3, Valor Esperado}
\end{axis}
\end{tikzpicture}
\caption{Ilustração da Lei dos Grandes Números: frequência relativa de caras em $n$ lançamentos de uma moeda equilibrada}
\end{figure}

\subsection{Teorema Central do Limite}

O Teorema Central do Limite estabelece que a soma de um grande número de variáveis aleatórias independentes e identicamente distribuídas segue aproximadamente uma distribuição normal, independentemente da distribuição original das variáveis.

Seja $X_1, X_2, \ldots, X_n$ uma sequência de variáveis aleatórias independentes e identicamente distribuídas com $E[X_i] = \mu$ e $\text{Var}(X_i) = \sigma^2 < \infty$. Defina $S_n = X_1 + X_2 + \ldots + X_n$. Então, para $n$ suficientemente grande:

\begin{equation}
\frac{S_n - n\mu}{\sigma\sqrt{n}} \xrightarrow{d} N(0, 1) \quad \text{quando } n \to \infty
\end{equation}

onde $\xrightarrow{d}$ denota convergência em distribuição.

\begin{tcolorbox}[colback=orange!5, colframe=orange!40, title=\textbf{Aplicação do Teorema Central do Limite}]
O TCL permite aproximar a distribuição da soma de várias variáveis aleatórias por uma distribuição normal, facilitando enormemente os cálculos de probabilidades. Por exemplo, para $n$ grande:

\begin{equation}
P\left(a \leq \frac{S_n - n\mu}{\sigma\sqrt{n}} \leq b\right) \approx \Phi(b) - \Phi(a)
\end{equation}

onde $\Phi$ é a função de distribuição acumulada da normal padrão.
\end{tcolorbox}

\begin{figure}[H]
\centering
\begin{tikzpicture}
\begin{axis}[
    width=10cm,
    height=6cm,
    xlabel={$x$},
    ylabel={Densidade},
    xmin=-4,
    xmax=4,
    ymin=0,
    ymax=0.7,
    grid=major,
    legend pos=north west,
    ]
    
    % Histograma para n=1 (distribuição uniforme)
    \addplot[ybar interval, fill=blue!20, draw=blue] coordinates {
        (-3, 0) (-2, 0) (-1, 0) (0, 0.5) (1, 0.5) (2, 0) (3, 0) (4, 0)
    };
    
    % Histograma para n=2 (soma de duas uniformes)
    \addplot[ybar interval, fill=green!20, draw=green] coordinates {
        (-3, 0) (-2, 0) (-1, 0.25) (0, 0.5) (1, 0.25) (2, 0) (3, 0) (4, 0)
    };
    
    % Histograma para n=12 (soma de doze uniformes)
    \addplot[thick, domain=-4:4, samples=100, red] {1/(sqrt(12/12)*sqrt(2*pi))*exp(-(x-0)^2/(2*(12/12)))};
    
    \legend{$n=1$, $n=2$, $n=12$}
\end{axis}
\end{tikzpicture}
\caption{Ilustração do Teorema Central do Limite: a soma de $n$ variáveis aleatórias independentes e identicamente distribuídas aproxima-se de uma distribuição normal conforme $n$ aumenta}
\end{figure}

\subsection{Aproximação da Distribuição Binomial pela Normal}

Para valores grandes de $n$ e valores de $p$ não muito próximos de 0 ou 1, a distribuição binomial $\text{Bin}(n, p)$ pode ser aproximada por uma distribuição normal com média $np$ e variância $np(1-p)$:

\begin{equation}
\text{Bin}(n, p) \approx N(np, np(1-p))
\end{equation}

Regra prática: a aproximação é considerada boa quando $np > 5$ e $n(1-p) > 5$.

\begin{figure}[H]
\centering
\begin{tikzpicture}
\begin{axis}[
    width=10cm,
    height=6cm,
    xlabel={$k$},
    ylabel={$P(X=k)$},
    xmin=0,
    xmax=40,
    ymin=0,
    ymax=0.15,
    grid=major,
    legend pos=north east,
    ]
    
    % Binomial(30, 0.5)
    \addplot[ybar, fill=blue!20, draw=blue, bar width=1pt] coordinates {
        (5, 0.000073) (6, 0.000244) (7, 0.000731) (8, 0.00198) (9, 0.00485) (10, 0.0107)
        (11, 0.0214) (12, 0.0389) (13, 0.0642) (14, 0.0962) (15, 0.1319)
        (16, 0.1649) (17, 0.1875) (18, 0.1941) (19, 0.1811) (20, 0.1527)
        (21, 0.1162) (22, 0.0804) (23, 0.0501) (24, 0.0282) (25, 0.0142)
    };
    
    % Normal approximation
    \addplot[thick, domain=0:40, samples=100, red] {1/(sqrt(30*0.5*0.5)*sqrt(2*pi))*exp(-((x-30*0.5)^2)/(2*30*0.5*0.5))};
    
    \legend{Binomial, Aproximação Normal}
\end{axis}
\end{tikzpicture}
\caption{Aproximação da distribuição binomial pela distribuição normal}
\end{figure}

\section{Estatística Inferencial Básica}

\subsection{Estimação Pontual}

A estimação pontual consiste em utilizar uma única estatística (estimador) para estimar um parâmetro desconhecido da população.

\begin{itemize}
    \item \textbf{Estimador}: função dos dados da amostra
    \item \textbf{Estimativa}: valor específico do estimador calculado a partir de dados amostrais
\end{itemize}

Propriedades desejáveis de um estimador:
\begin{itemize}
    \item \textbf{Não-viesado}: $E[\hat{\theta}] = \theta$ (o valor esperado do estimador é igual ao parâmetro)
    \item \textbf{Consistente}: $\hat{\theta} \xrightarrow{P} \theta$ quando $n \to \infty$ (o estimador converge em probabilidade para o parâmetro)
    \item \textbf{Eficiente}: tem a menor variância possível entre os estimadores não-viesados
\end{itemize}

Estimadores comuns:
\begin{align}
\text{Média amostral: } \overline{X} &= \frac{1}{n}\sum_{i=1}^{n}X_i \\
\text{Variância amostral: } S^2 &= \frac{1}{n-1}\sum_{i=1}^{n}(X_i - \overline{X})^2 \\
\text{Proporção amostral: } \hat{p} &= \frac{X}{n} 
\end{align}

\subsection{Intervalos de Confiança}

Um intervalo de confiança é um intervalo de valores que, com uma certa probabilidade (nível de confiança), contém o verdadeiro valor do parâmetro populacional.

Para a média populacional $\mu$ de uma distribuição normal com variância $\sigma^2$ conhecida, um intervalo de confiança de nível $(1-\alpha)$ é:

\begin{equation}
\overline{X} \pm z_{\alpha/2} \frac{\sigma}{\sqrt{n}}
\end{equation}

onde $z_{\alpha/2}$ é o quantil de ordem $1-\alpha/2$ da distribuição normal padrão.

Se a variância $\sigma^2$ é desconhecida e a amostra é de uma distribuição normal, utiliza-se a distribuição t de Student:

\begin{equation}
\overline{X} \pm t_{\alpha/2, n-1} \frac{S}{\sqrt{n}}
\end{equation}

onde $t_{\alpha/2, n-1}$ é o quantil de ordem $1-\alpha/2$ da distribuição t com $n-1$ graus de liberdade.

\begin{tcolorbox}[colback=blue!5, colframe=blue!40, title=\textbf{Interpretação Correta de um Intervalo de Confiança}]
Um intervalo de confiança de 95\% não significa que a probabilidade do parâmetro estar no intervalo é 95\%. Significa que, se muitas amostras fossem coletadas e intervalos de confiança calculados, 95\% desses intervalos conteriam o verdadeiro valor do parâmetro.
\end{tcolorbox}

\subsection{Testes de Hipóteses}

Um teste de hipóteses é um procedimento para decidir se uma hipótese sobre a população é plausível, com base em dados amostrais.

\begin{itemize}
    \item \textbf{Hipótese nula} ($H_0$): hipótese que está sendo testada
    \item \textbf{Hipótese alternativa} ($H_1$ ou $H_a$): hipótese que será aceita se $H_0$ for rejeitada
    \item \textbf{Estatística de teste}: função dos dados que é usada para decidir se $H_0$ deve ser rejeitada
    \item \textbf{Região crítica}: conjunto de valores da estatística de teste que levam à rejeição de $H_0$
    \item \textbf{Nível de significância} ($\alpha$): probabilidade de rejeitar $H_0$ quando ela é verdadeira (erro Tipo I)
    \item \textbf{Valor-p}: probabilidade de observar um valor da estatística de teste pelo menos tão extremo quanto o observado, supondo que $H_0$ seja verdadeira
\end{itemize}

Para testar $H_0: \mu = \mu_0$ contra $H_1: \mu \neq \mu_0$ (teste bilateral) para uma distribuição normal com $\sigma$ conhecido:

\begin{equation}
Z = \frac{\overline{X} - \mu_0}{\sigma/\sqrt{n}}
\end{equation}

Rejeita-se $H_0$ ao nível de significância $\alpha$ se $|Z| > z_{\alpha/2}$.

\begin{figure}[H]
\centering
\begin{tikzpicture}
\begin{axis}[
    width=10cm,
    height=6cm,
    xlabel={$z$},
    ylabel={$f(z)$},
    xmin=-4,
    xmax=4,
    ymin=0,
    ymax=0.5,
    grid=major,
    ]
    
    % Normal padrão
    \addplot[thick, domain=-4:4, samples=100, blue] {1/sqrt(2*pi)*exp(-x^2/2)};
    
    % Região crítica (bilateral)
    \addplot[fill=red!20, domain=-4:-1.96, samples=100] {1/sqrt(2*pi)*exp(-x^2/2)} \closedcycle;
    \addplot[fill=red!20, domain=1.96:4, samples=100] {1/sqrt(2*pi)*exp(-x^2/2)} \closedcycle;
    
    % Linhas verticais
    \draw[dashed] (axis cs:-1.96,0) -- (axis cs:-1.96,0.4);
    \draw[dashed] (axis cs:1.96,0) -- (axis cs:1.96,0.4);
    
    % Texto
    \node at (axis cs:0,0.45) {Região de Aceitação};
    \node at (axis cs:-3,0.1) {Região Crítica};
    \node at (axis cs:3,0.1) {Região Crítica};
    \node at (axis cs:-1.96,-0.02) {$-z_{\alpha/2}$};
    \node at (axis cs:1.96,-0.02) {$z_{\alpha/2}$};
\end{axis}
\end{tikzpicture}
\caption{Regiões de aceitação e crítica para um teste bilateral ao nível de significância $\alpha = 5\%$}
\end{figure}

\section{Exemplo de Aplicação Completa}

\subsection{Contexto do Problema}

Suponha que uma empresa de semicondutores quer testar se um novo método de fabricação melhora a vida útil dos seus chips. A vida útil dos chips com o método atual segue aproximadamente uma distribuição normal com média $\mu = 1200$ horas e desvio padrão $\sigma = 150$ horas. 

A empresa testa 36 chips produzidos com o novo método e obtém uma vida útil média de 1245 horas. Eles querem saber se há evidência estatística de que o novo método melhora a vida útil.

\subsection{Formulação do Teste de Hipóteses}

\begin{align}
H_0: \mu &= 1200 \text{ (o novo método não melhora a vida útil)} \\
H_1: \mu &> 1200 \text{ (o novo método melhora a vida útil)}
\end{align}

\subsection{Cálculo da Estatística de Teste}

Considerando o desvio padrão conhecido:

\begin{equation}
Z = \frac{\overline{X} - \mu_0}{\sigma/\sqrt{n}} = \frac{1245 - 1200}{150/\sqrt{36}} = \frac{45}{25} = 1.8
\end{equation}

\subsection{Decisão}

Para um teste unilateral com nível de significância $\alpha = 0.05$, o valor crítico é $z_{0.05} = 1.645$.

Como $Z = 1.8 > 1.645$, rejeitamos $H_0$ ao nível de significância de 5\%.

\subsection{Cálculo do Valor-p}

\begin{equation}
\text{Valor-p} = P(Z > 1.8) = 1 - \Phi(1.8) \approx 0.0359
\end{equation}

onde $\Phi$ é a função de distribuição acumulada da normal padrão.

\subsection{Interpretação}

O valor-p de 0.0359 é menor que o nível de significância de 0.05, o que confirma nossa decisão de rejeitar $H_0$. Há evidência estatística para concluir que o novo método de fabricação melhora a vida útil dos chips.

\subsection{Intervalo de Confiança}

Um intervalo de confiança de 95\% para a média da vida útil com o novo método é:

\begin{align}
\overline{X} \pm z_{0.025} \frac{\sigma}{\sqrt{n}} &= 1245 \pm 1.96 \cdot \frac{150}{\sqrt{36}} \\
&= 1245 \pm 1.96 \cdot 25 \\
&= 1245 \pm 49 \\
&= [1196, 1294]
\end{align}

\subsection{Interpretação do Intervalo}

O intervalo de confiança de 95\% para a média da vida útil dos chips produzidos com o novo método vai de 1196 a 1294 horas. Como este intervalo contém o valor 1200 (embora por uma margem pequena), isso sugere que a melhoria pode não ser tão substancial quanto se esperava inicialmente.

\subsection{Conclusão}

Embora haja evidência estatística para concluir que o novo método melhora a vida útil dos chips (teste de hipóteses), o intervalo de confiança sugere que a melhoria pode ser modesta. A empresa deve considerar outros fatores, como o custo da implementação do novo método, antes de tomar uma decisão final.

\begin{tcolorbox}[colback=green!5, colframe=green!40, title=\textbf{Dica de Interpretação}]
Sempre considere tanto o resultado do teste de hipóteses quanto o intervalo de confiança ao tomar decisões baseadas em dados estatísticos. O teste de hipóteses fornece uma decisão binária (rejeitar ou não rejeitar), enquanto o intervalo de confiança fornece uma gama de valores plausíveis para o parâmetro, o que pode ser mais informativo para a tomada de decisões práticas.
\end{tcolorbox}

\end{document}